% !TeX spellcheck = nl_BE
\documentclass{article}
\usepackage{hyperref}
\usepackage{graphicx}
\usepackage{listings}
\newcommand{\HRule}{\rule{\linewidth}{0.5mm}}
\newcommand{\thedate}{21 Mei 2014}
\newcommand{\projectname}{ProjectGedistribueerdeSystemen-ArnoDeWitte-15_05_214}
\title{Codeverslag Project Gedistribueerde Systemen}
\author{Arno De Witte\\
Vrije Universiteit Brussel}
\date{21 Mei 2014}
\begin{document}


\begin{titlepage}
\begin{center}

\includegraphics[width=0.60\textwidth]{./VUB_logo_compact.jpg}~\\[1cm]


\textsc{\Large Gedistribueerde Systemen}\\[0.5cm]

% Title
\HRule \\[0.4cm]
{ \huge \bfseries Codeverslag Project}\\[0.4cm]

\HRule \\[1.5cm]

% Author and supervisor
\begin{minipage}{0.4\textwidth}
\begin{flushleft} \large
Arno \textsc{De Witte}\\
\end{flushleft}
\end{minipage}
\begin{minipage}{0.5\textwidth}
\begin{flushright} \large
\emph{Docent:}\\ Prof. Dr. Tom Van Cutsem\\
\emph{Assistent:}\\ Laure Philips
\end{flushright}
\end{minipage}

\vfill

% Bottom of the page
{\large \thedate}

\end{center}
\end{titlepage}

%\maketitle
\newpage
\tableofcontents
\newpage


\section{Inleiding}\label{inleiding}
De opdracht van het project voor het vak Gedistribueerde Systemen was om een versiebeheersysteem te maken. Er moesten enkele basis commando's worden ge\"{i}mplementeerd waaronder checkout, commit en andere. Dit moest gebeuren in Java en op het TCP/IP protocol. Er moest een eigen protocol worden ontwikkelt om tussen client en server te kunnen communiceren.

\section{Testen}\label{test}
\subsection{Compileren}
Om de bestanden te compileren moet je het volgende commando uitvoeren:
\lstinputlisting{./compile.sh}
Of je kan de meegeleverde file \emph{compile.sh} uitvoeren. Dit zal in de folder waarin je de commando's uitvoert of waarin je het script uitvoert, klasse bestanden genereren.
\subsection{Uitvoeren}
Om de client uit te voeren moet je 


\end{document}
